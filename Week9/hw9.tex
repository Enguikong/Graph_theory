\documentclass{article}
\usepackage{xeCJK,amsmath,geometry,float,graphicx,amssymb,zhnumber,booktabs,setspace,tasks,verbatim,amsthm,amsfonts,mathdots}
\usepackage{listings,xcolor}
\geometry{a4paper,scale=0.8}  
\renewcommand\arraystretch{2}
\title{图论作业 (第九周)}
\author{PB20000113孔浩宇}
\begin{document}
\maketitle
\section*{Ch5}
\subsection*{11.}
\begin{proof}
        取$S\subseteq X$,则$(X-S)\cup N(S)$为$G$的一个覆盖.任意$G$的覆盖$C$,记$C\cap X=C_X$,若
    \[
    \exists\ u\in (X-C_X) \cup N(C_X) ,\ u\notin\ C
    \ \Rightarrow\ 
    \exists\ e=pu\ (e\in E(G)),\ p,u\notin C
    \ \Rightarrow\ 
        \mbox{矛盾}.
    \]
    即覆盖$C$可以写成$(X-S)\cup N(S)\ (S\subseteq X)$,又
    \[
        G\mbox{为二分图}\xrightarrow{\mbox{定理5.2}}
        \alpha(G)=\beta(G).
    \]
    有
    \begin{align*}
        \alpha(G)&=\beta(G)\\
        &=\min |C|\\
        &=\min\limits_{S\subseteq X} | (X-S)\cup N(S) |\\
        &=\min\limits_{S\subseteq X} \left(|X-S|+|N(S)| \right)\\
        &=\min\limits_{S\subseteq X} \left(|X|-|S|+ |N(S) |\right)\\
        &=|X|-\max\limits_{S\subseteq X} \left(|S|-|N(S)|\right).
    \end{align*}
\end{proof}

\subsection*{13.}
\begin{proof}
    对于二分图$G=(X,E,Y)$,增加边使得$Y$为完全图,记此时的图为$H$.则$G$中存在将$X$都许配的匹配等价于$H$有完备匹配.
    $\forall\ S\subseteq X,\ N_{H} (S)=N_{G} (S)$.
    \begin{enumerate}
        \item [(1)]必要性:$G$中存在将$X$都许配的匹配.
        \[
            \begin{cases}
                \forall\ S\subseteq X & \xrightarrow{Tutte\mbox{定理}} o(H-N_H (S))\leq |N_H (S)|\\
                \\
                \mbox{在}H-N_H (S)\mbox{中},&\forall\ s\in S\mbox{为孤立点}\ \Rightarrow\ |S|\leq o(H- N_H (S))
            \end{cases}
            \ \Rightarrow\ 
            |S|\leq |N_H (S)|=|N_G (S)|.
        \]
        \item [(2)]充分性:$\forall\ S\subseteq X,\ |S|\leq |N(S)|$.
        \[
            \forall\ V\subseteq\ V(H),\ S=V\cap X,\ W=V\cap Y.
            \xrightarrow{Y\mbox{为完全图}}
            H-S\mbox{至多比}H\mbox{多一个奇片}.
        \]
        \[
            \mbox{记}H-W\mbox{中}X\mbox{中的孤立点集合为}S'
            \ \Rightarrow\ 
            |N(S')|\leq |W|
            \ \Rightarrow\ 
            |S'|\leq |N(S')|\leq |W|.
        \]
        \begin{enumerate}
            \item []若$|S'|=|W|$,则$o(H-W)=|S'|=|W|$.
            \item []若$|S'|\leq |W|-1$,则$o(H-W)\leq |S'|+1\leq |W|$.
            \item []若$|S|$为偶数,则$o(H-V)=o(H-W)\leq |W|\leq |V|$.
            \item []若$|S|$为奇数,则$o(H-S)=o(H-W)+1\leq |W|+1\leq |V|$.
        \end{enumerate}
        综上,$\forall\ S\subseteq\ V(H),\ o(H-S)\leq |S|$,由Tutte定理可得,H有完备匹配,即$G$中存在将$X$都许配的匹配.
    \end{enumerate}
\end{proof}
\subsection*{14.}
\begin{proof}
    \begin{enumerate}
        \item []
        \item []$\forall\ W\subseteq V(G)$,记$G-W$中的奇片为$G_1,\ldots,G_p,\ q_i\ (1\leq i\leq p)=|E_i|,\ E_i=\{uv|\ u\in G_i,q\in W\}$.
        \[
            q_i =\sum\limits_{v\in V(G_i)} \deg(v) -2|E(G_i)|=k|V(G_i)|-2|E(G_i)|
            \xrightarrow{|V(G_i)|\mbox{为奇数}}
            q_i\mbox{与}k\mbox{奇偶性相同}.
        \]
        又$G$是$k-1$边连通的,在删去$G_i$与$W$之间的边后$G$不再连通,有
        \[
            q_i\geq k-1,\mbox{又}q_i \mbox{与} k\mbox{奇偶性相同}
            \ \Rightarrow\ 
            q_i\geq k.
        \]
        于是
        \[
            \sum\limits_{i=1}^{p} q_i \geq kp
            \ \Rightarrow\ 
            p\leq \displaystyle{\frac{1}{k}}\sum\limits_{i=1}^{p} q_i
            \leq \displaystyle{\frac{1}{k}}  \deg(v)= |W|.
        \]
        若$W=\phi$,由于$\nu(G)$为偶数,此时$p=0=|W|,p\leq |W|$仍成立.即
        \[
            \forall\ S\subseteq V(G),\ o(G-S)\leq |S|
            \ \Rightarrow\ \
            G\mbox{有完备匹配}.
        \]
    \end{enumerate}
\end{proof}
\subsection*{16.}
    \begin{table}[!ht]
        \centering
        \begin{tabular}{|c|c|c|c|c|}
        \hline
            一 & 二 & 三 & 四 & 五  \\ \hline
            a & b & c & e & d,f  \\ \hline
            a & b & c,d & e & f  \\ \hline
            a & b & d & c & e,f  \\ \hline
            a & b & d & c,e & f  \\ \hline
            a & b,d & c & e & f  \\ \hline
            a & b,c & d & e & f  \\ \hline
            a,e & b & d & c & f  \\ \hline
            a,f & b & c & e & d  \\ \hline
            a,f & b & d & c & e  \\ \hline
        \end{tabular}
    \end{table}

\subsection*{19.}
\begin{proof}
    \begin{enumerate}
        \item []取$x_i\in X,\ y_i\in Y$.
        \item [(1)]$x_i\in S,\ y_i\in T$.
        \begin{align*}
            \widehat{l}(x_i)+\widehat{l}(y_i)
            &= l(x_i)-\alpha_l+l(y_i)+\alpha_l\\
            &= l(x_i)+l(y_i)\\
            &\geq w(x_i y_i).
        \end{align*}

        \item [(2)]$x_i\in S,\ y_i\notin T$.
        \begin{align*}
            \widehat{l}(x_i)+\widehat{l}(y_i)
            &= l(x_i)-\alpha_l +l(y_i)\\
            &= l(x_i)+l(y_i)- \min\limits_{x\in X,y\notin Y} \{l(x)+l(y)-w(xy) \}\\
            &\geq l(x_i)+l(y_i)- \left[l(x_i)+l(y_i)-w(x_i y_i) \right] \\
            &=w(x_i y_i).
        \end{align*}

        \item [(3)]$x_i\notin S,\ y_i\in T$.
        \begin{align*}
            \widehat{l}(x_i)+\widehat{l}(y_i)
            &= l(x_i)+l(y_i)+\alpha_{l} \\
            &\geq l(x_i)+l(y_i)\\
            &\geq w(x_i y_i).
        \end{align*}

        \item [(4)]$x_i\notin S,\ y_i\notin T$.
        \begin{align*}
            \widehat{l}(x_i)+\widehat{l}(y_i)
            &= l(x_i)+l(y_i)\\
            &\geq w(x_i y_i).
        \end{align*}
    \end{enumerate}
\end{proof}

\subsection*{20.}
\begin{proof}
    \begin{enumerate}
        \item []取$x_i\in X,\ y_i\in Y$.
        \item [(1)]$x_i\in S,\ y_i\in T$.
        \begin{align*}
            \widehat{l}(x_i)+\widehat{l}(y_i)
            &= l(x_i)-\alpha_l+l(y_i)+\alpha_l\\
            &= l(x_i)+l(y_i)
        \end{align*}
        修改前后顶点子集不变.

        \item [(2)]$x_i\in S,\ y_i\notin T$.
        \begin{align*}
            \widehat{l}(x_i)+\widehat{l}(y_i)
            &= l(x_i)-\alpha_l +l(y_i)\\
            &= l(x_i)+l(y_i)- \min\limits_{x\in X,y\notin Y} \{l(x)+l(y)-w(xy) \}\\
            &\geq l(x_i)+l(y_i)- \left[l(x_i)+l(y_i)-w(x_i y_i) \right] \\
            &=w(x_i y_i).
        \end{align*}
        且至少存在一对$(x_i,y_i)$使等号成立 (当$x_i,y_i$使得$l(x)+l(y)-w(xy)$最小时),
        故把$Y-T$中至少$1$个顶点移入$T$中.

        \item [(3)]$x_i\notin S,\ y_i\in T$.
        \begin{align*}
            \widehat{l}(x_i)+\widehat{l}(y_i)
            &= l(x_i)+l(y_i)+\alpha_{l} \\
            &\geq l(x_i)+l(y_i)\\
            &\geq w(x_i y_i).
        \end{align*}
        修改前后顶点子集不变.

        \item [(4)]$x_i\notin S,\ y_i\notin T$.
        \begin{align*}
            \widehat{l}(x_i)+\widehat{l}(y_i)
            &= l(x_i)+l(y_i)
        \end{align*}
        修改前后顶点子集不变.
    \end{enumerate}
    综上,算法第三步修改顶标后,顶点子集元素至少增加一个.
\end{proof}

\section*{Ch6}
\subsection*{3.}
\begin{proof}
    \begin{enumerate}
        \item []对$k$进行归纳。
        \item [(1)]$k=1$时,连通图$G$有2个奇度顶点,则由推论$6.1$可知$G$存在$Euler$迹$C$,显然有
        \[
            E(G)=E(C).    
        \]
        即$k=1$时结论成立。
        \item [(2)]设$k=n$时结论成立.则$k=n+1$时,记$G$中奇度顶点集合为$S$,取$u,v\in S$,令
        \[
            G'=G\cup uv\ (uv\mbox{为新增的边,重边也无所谓})    
        \]
        对于图$G'$,奇度顶点集$S'=S-{u,v},\ |S'|=|S|-2=2n$,由于$k=n$时结论成立,即
        \[
            G\mbox{中存在}k\mbox{条不重的行迹}P_1,P_2,\ldots,P_n,\mbox{使得}E(G)=\bigcup_{i=1}^{n}E(P_i).
        \]
        不妨设$P_m$中含新增的边$uv$,若$P_m -uv$为G中两条不重的行迹,则记为$P_{n+1},P_{n+2}$,
        若$P_m-uv$为一条行迹,则分成两条不重的行迹$P_{n+1},P_{n+2}$,此时有
        \[
            E(G)=P_1 \cup \cdots \cup P_{m-1} \cup P_{m+1}\cup \cdots P_{n} \cup P_{n+1}\cup P_{n+2}.
        \]
        又$P_i\ (1\leq i\leq m-1,m+1\leq i\leq n+2)$不重,即证$k=n+1$时结论也成立.
    \end{enumerate}
    综上,即证。
\end{proof}
\subsection*{5.}
定义顶点
\[
    V(G)=\mbox{所有由}\alpha,\beta,\gamma\mbox{组成的不重复的三位符号}
\]
定义边
\[
    E(G)=\{\overrightarrow{uv} |\ u,v\in V(G),u\mbox{可以通过左移之后加上一个字母得到}v\}
\]
以$V(G),E(G)$构建有向图$G$.则$\forall\ v\in V(G),\ \deg^{+}(v)=\deg^{-}(v)=3$,
由定理6.2可得$G$为$Euler$图,可根据$Euler$回路构造队列.如下为其中一条
\[
    \rightarrow 
    \alpha\ \beta\ \alpha\ \alpha\ \alpha\ \gamma\  
    \alpha\ \beta\ \gamma\ \beta\ \beta\ \beta\ 
    \alpha\ \beta\ \beta\ \gamma\ \gamma\ \gamma\
    \alpha\ \gamma\  \gamma\ \beta\ \gamma\ \alpha\
    \rightarrow
\]

\end{document}