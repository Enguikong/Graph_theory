\documentclass{article}
\usepackage{xeCJK,amsmath,geometry,float,graphicx,amssymb,zhnumber,booktabs,setspace,tasks,verbatim,amsthm,amsfonts,mathdots}
\usepackage{listings,xcolor,diagbox}
\geometry{a4paper,scale=0.8}  
\renewcommand\arraystretch{2}
\title{图论作业 (Week 11 \& Week 12)}
\author{PB20000113孔浩宇}
\begin{document}
\maketitle
\section*{Ch7}
\subsection*{2.}
\begin{proof}
    将图$G$分为$\chi (G)$个内部无连边的子图,记为划分$C= (V_1,\ldots,V_{\chi (G)}),\ v_i=|V_i|$.则
    \[
        2\varepsilon 
        \leq \sum\limits_{i=1}^{\chi (G)} v_i \cdot (\nu - v_i)
        =\sum\limits_{i=1}^{\chi (G)} v_i \nu - \sum\limits_{i=1}^{\chi (G)} v_i^2
        =\nu ^2 - \sum\limits_{i=1}^{\chi (G)} v_i^2.
    \]
    又有
    \[
        \chi(G) \cdot \sum\limits_{i=1}^{\chi (G)} v_i^2
        \geq {\left(\sum\limits_{i=1}^{\chi (G)} v_i \right)}^2
        \ \Rightarrow\ 
        \nu ^2 -\sum\limits_{i=1}^{\chi (G)} v_i^2
        \leq 2 \varepsilon \leq \nu ^2 \left(1-\displaystyle{\frac{1}{\chi (G)}}\right)
    \]
    可得
    \[
        \chi (G) \geq \displaystyle{\frac{\nu ^2}{\nu ^2 -2\varepsilon }}.
    \]
\end{proof}

\subsection*{4.}
\begin{proof}
    不妨设$\chi (G)\geq 6$,则有无交划分$C= (V_1 ,\ V_2,\ \ldots ,\ V_k)\ (k\geq 6)$.取顶点导出子图
    \[
        \begin{cases}
            \ G_1&= G[V_1 \cup V_2 \cup V_3]\\
            \\
            \ G_2&= G[V_4 \cup V_5 \cup V_6]
        \end{cases}
        \ \Rightarrow\ 
        \chi(G_1)=\chi(G_2)=3
        \ \Rightarrow\ 
        G_1, G_2\mbox{均含奇圈}
    \]
    又
    \[
        V(G_1) \cap V(G_2) =\phi
        \ \Rightarrow\ 
        \mbox{矛盾!}
    \]
    即证$\chi (G)\leq 5$.
\end{proof}

\subsection*{6.}
\begin{proof}
    记$G$的最小着色划分为$C= (V_1 ,\ V_2,\ \ldots ,\ V_{\chi (G)})$,
    不妨设$\max \{|V_i|\ 1\leq i\leq \chi (G) \} = |V_k|=v_k$.则
    \[
        v_k \leq \nu -(\chi (G)-1)
    \]
    又对于图$G^c$,$\{uv |\ u\in V_i,\ v\in V_j,\ 1\leq i\neq j\leq \chi(G) \}=\phi$,存在着色
    \[
        C={(M_1,\ M_2,\ \ldots,\ M_{v_k})},\ Card [M_i\cap V_j]\leq 1
        \ \Rightarrow\ 
        \chi (G^c) = v_k 
    \]
    可得
    \[
        \chi (G) +\chi (G^c) 
        \leq \chi (G) +\nu -(\chi (G) -1)
        =\nu + 1.
    \]
\end{proof}

\subsection*{8.}
    记该图为$G$,中心顶点编号为$v_0$,
    圈上的顶点按顺时针依次为$v_1,v_2,···,v_{\nu-1}$。显然
    \[
        \chi '(G)\geq \Delta(G) =\deg (v_0)=\nu -1.
    \]

    考虑下面的$\nu-1$着色方法:
    \[
        \begin{cases}
            &v_0 \mbox{与} v_i \mbox{之间着第$i$色} (1\leq i\leq \nu-1)\\
            \\
            &v_i \mbox{与} v_{i+1} \mbox{之间着第} i+2 \mbox{(模$k$意义下)色}
        \end{cases}
        \ \Rightarrow\ 
        \mbox{可知其为$G$的正常着色}
        \ \Rightarrow\ 
        \chi '(G)\leq \nu - 1.
    \]
    即
    \[
        \chi '(G)=\nu -1.
    \]
\subsection*{14.}
\begin{enumerate}
    \item []根据矩阵$A = {(a_{ij} )}_{4×5}$构造二分图
    \[
        G = (X, E, Y),\ 
        X =\{ x_1, x_2, x_3, x_4 \},
        \ Y = \{y_1, y_2, y_3, y_4, y_5\},
        \ E = \{x_i y_j |\ a_{ij}\geq 0 \}.
    \]
    \item [(1)]最少需要$\Delta=4$个课时。
    \item [(2)]有$\lceil\frac{\varepsilon}{\Delta}\rceil =\lceil\frac{13}{4}\rceil=4$,最少需要4间教室.
    课表安排如图 (同时满足 (1) (2) 要求)
    \begin{table*}[htbp]
        \centering
        \begin{tabular}[]{c|cccc}
        \diagbox{教师}{课时} & 1 & 2 & 3 & 4 \\
        \hline
        $x_1$ & $y_1$ & $ - $ & $y_3$ & $ - $ \\
        $x_2$ & $y_3$ & $y_4$ & $y_1$ & $ - $ \\
        $x_3$ & $y_2$ & $y_3$ & $y_4$ & $y_5$ \\
        $x_4$ & $y_4$ & $y_5$ & $y_5$ & $y_2$ 
    \end{tabular}
    \end{table*}
    
    
\end{enumerate}

\subsection*{17.}
\begin{proof}
    \[
        G\mbox{为极大平面图}
        \Rightarrow
        \begin{cases}
            & G\mbox{每个面均为三角形}\\
            \\
            & G\mbox{无环}, G^*\mbox{无桥}
        \end{cases}
        \Rightarrow
        \begin{cases}
            \ G^* & \mbox{为3次正则图}\\
            \\
            \ G^* & \mbox{为2-边连通的}
        \end{cases}
    \]
\end{proof}

\clearpage
\section*{Ch8}
\subsection*{1.}
    不妨设顶点为$v_1, v_2 , v_3 , v_4 , v_5$.
    \begin{enumerate}
        \item [(1)]定向与$v_1$关联且未定向的边
        \[
            \mbox{共有4条,且等价,故共有5种定向方式}
        \]
        \item [(2)]定向与$v_2$关联且未定向的边
        \[
            \mbox{共有3条,且等价,故共有4种定向方式}
        \]
        \item [(3)]定向与$v_3$关联且未定向的边
        \[
            \mbox{共有2条,且等价,故共有3种定向方式}
        \]
        \item [(4)]定向与$v_4$关联且未定向的边
        \[
            \mbox{共有1条,故共有2种定向方式}
        \]
        \item [(5)]不存在与$v_5$关联且未定向的边,故共有1种定向方式.
        \item [(6)]又$v_1, v_2 , v_3 , v_4 , v_5$等价,故
        \[
            \mbox{共有定向方式}
            \displaystyle{\frac{5\times 4\times 3\times 2\times 1}{5}}
            =24\mbox{种}.
        \]
    \end{enumerate}

\subsection*{2.}
\begin{proof}
    \begin{enumerate}
        \item []记$|V(D)|=\nu $.
        \item [(1)]假设$\delta^-\geq 1$,则$\forall\ v\in V(D)$,都有$\deg^- (v)\geq 1$。
        取$v_1 \in V(D)$,令$S_1 ={v_1}$,则
        \[
            \mbox{存在$v_2$使得$(v_2,v_1)\in E(D)$,且$v_2\notin S_1$(否则有有向圈).}
        \]
        
        令$S_{2}=S_1 \bigcup{v_2}$,再找到$v_3$使得$(v_3,v_2)\in E$,且$v_3 \notin S_2$。
        以此类推,
        \[
            \forall\ S_\nu = V(D) =\{ v_1,v_2,···,v_\nu \},\ 
            \exists\ v_{\nu+1}\notin S_\nu\ \mbox{且}\ (v_{\nu+1},v_{\nu})\in E(D),\ 
            \mbox{显然矛盾.}
        \]
        故有$\delta^-=0$。
        \item [(2)]
        \begin{enumerate}
            \item []
            \item [(a)]先证$\delta ^+ =0$
            \item []假设$\delta^+ \geq 1$,则$\forall\ v\in V(D)$,都有$\deg^+ (v)\geq 1$。
            取$v_1 \in V(D)$,令$S_1 ={v_1}$,则
            \[
                \mbox{存在$v_2$使得$(v_1,v_2)\in E(D)$,且$v_2\notin S_1$(否则有有向圈).}
            \]
            
            令$S_{2}=S_1 \bigcup{v_2}$,再找到$v_3$使得$(v_2,v_3)\in E$,且$v_3 \notin S_2$。
            以此类推,
            \[
                \forall\ S_\nu = V(D) =\{ v_1,v_2,···,v_\nu \},\ 
                \exists\ v_{\nu+1}\notin S_\nu\ \mbox{且}\ (v_{\nu},v_{\nu+1})\in E(D),\ 
                \mbox{显然矛盾.}
            \]
            故有$\delta^+=0$。
            \item [(b)]记$D=D_1$由$\delta ^+ (D_1)=0$可得$\exists\ v_1\in D_1,\ \deg ^+ (v_1)=0$,
            取$D_2=D_1 -v_1$.
            \[
                D_2\subseteq D_1\ \Rightarrow\ 
                D_2 \mbox{不含有向圈}\ \Rightarrow\ 
                \exists\ v_2\in D_2,\ \deg^+ (v_2)=0.
            \]
            以此类推,可得顶点序列$v_1,\ v_2,\ldots,\ v_{\nu}$,且满足要求.
        \end{enumerate}
    \end{enumerate}
\end{proof}

\subsection*{3.}
\begin{proof}
    \begin{enumerate}
        \item []
        \item []
        由奇度顶点为偶数,不妨设$G$中奇度顶点为$v_1,v_2,···,v_{2k}\ (k\in N)$,则
        \[
            \mbox{取}E=\{v_i v_{i+k}|\ 1\leq i\leq k \},\ G'=G+E
            \mbox{为欧拉图,}
        \]
        \item []
        $G'$中存在一条欧拉回路,沿着回路给图中每条边定向得到有向图$D'$,显然有
        \[
            \forall\ v \in V,\ 
            \deg_{D'}^+ (v)=\deg_{D'} ^- (v).
        \]
        \item []
        取$G$的一个定向图$D=D'-E$,从$D'$到$D$,每个顶点关联的边至多减少1,故
        \[
            \forall\ v\in V,\ 
            |\deg_{D}^+ (v)-\deg_{D} ^-(v)|\leq 1.
        \]
    \end{enumerate}
\end{proof}

\end{document}